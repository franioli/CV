%------------------------------------------------------------------------
%	EXPERIENCE
%------------------------------------------------------------------------

%  My research focuses on photogrammetry and 3D reconstruction for territorial and infrastructural monitoring. 
%  My research focuses on UAV-based and terrestrial photogrammetry for land and infrastructure monitoring. I am working on a low-cost image-based system for 4D glacier monitoring (\href{https://github.com/franioli/icepy4d}{ICEpy4D}) and UAV-photogrammetry for structural health assessment and crack detection on concrete bridges.
%  I am contributing to \href{https://github.com/3DOM-FBK/deep-image-matching}{Deep-Image-Matching]} a multiview matching with deep-learning and hand-crafted local features SfM 

\cvsection{Experience}

\cvevent{06/2025 - now}{Post-doc}{CNR IRPI}{
Post-doc in the  Institute for Geo-Hydrological Protection (IRPI) of the Italian National Research Council (CNR). 
}

\cvevent{10/2024 - 05/2025}{Post-doc}{University of Zurich, Dept. of Geography}{
Post-doc in the \href{https://www.geo.uzh.ch/en/units/rse.html}{Remote Sensing of Environmental Changes} group. I developed automated pipelines for large-scale DEM reconstruction using satellite multi-view stereo for geodetic mass balance.
}

\cvevent{11/2020 - 07/2024}{PhD (Cum Laude)}{Politecnico di Milano, DICA (IT)}{
\href{https://hdl.handle.net/10589/224092}{PhD Thesis}: \textit{Multi-temporal and Multi-scale photogrammetry for Alpine Glacier Monitoring}.\\
Development of novel photogrammetric techniques for territorial and infrastructural monitoring. 
Created \href{https://github.com/franioli/icepy4d}{ICEpy4D}{,} an image-based pipeline for 4D glacier monitoring with low-cost stereo cameras.  
Contributed to \href{https://github.com/3DOM-FBK/deep-image-matching}{Deep-Image-Matching}{,} a multi-view image matching library with deep learning for SfM.
Application of UAV photogrammetry for structural health assessment{,} including crack detection on concrete bridges.
}

 \cvevent{04/2022-07/2022}{Visiting PhD student}{University of Twente, ITC (NL)}{
Development of a deep learning wide-baseline stereo matching workflow for 4D monitoring of an alpine glacier with low-cost time-lapse cameras. 
\href{https://doi.org/10.1007/s41064-023-00272-w}{[Paper]} \href{https://github.com/franioli/icepy4d}{[Code]}
}

\cvevent{2020 - 2024}{Teaching Assistant}{Politecnico di Milano}{
\textit{Photogrammetry and UAV surveying} (MSc) [2024],
\textit{Trattamento delle Osservazini} - Statistics (BSc) [2020 - 2021 - 2022 - 2023],
\textit{Sistemi Informativi Territoriali} - GIS (BSc) [2020 - 2021],
\textit{Tecniche di rilievo e modellazione 3D per l'architettura} - 3D modelling for architecture (BSc) [2020 - 2021 - 2022].
}

\cvevent{2021 - 2024}{Tutoring in Summer Schools}{Politecnico di Milano}{\textit{Design and Execution of Topographic Surveys for Land Monitoring} @ Belvedere Glacier (IT) aimed at introducing BSc and MSc students to topographic fieldwork in mountain environments. 
}

\cvevent{2022}{Topographic technical consultant}{Prof. Alberto Bianchi}{Topographic consultant for the Technical Consultant of Office and Part (CTU) R.G. 717/2019}

\cvevent{2022}{Topographic technician}{Gini Telecom}{UAV surveys for telecommunication antennas}