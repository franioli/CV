%=======================================================================%
%   DOCUMENT DEFINITION
%=======================================================================%
\documentclass[10pt,a4paper]{article}

%------------------------------------------------------------------------
%   ENCODING + FONT
%------------------------------------------------------------------------
\usepackage[utf8]{inputenc}
\usepackage[T1]{fontenc}
\usepackage[default]{gillius}
\renewcommand*\familydefault{\sfdefault}
\usepackage{moresize}
\usepackage{fontawesome5}

%------------------------------------------------------------------------
%   LOGIC
%------------------------------------------------------------------------
\usepackage{xifthen}

%------------------------------------------------------------------------
%   PAGE LAYOUT
%------------------------------------------------------------------------
\usepackage[a4paper]{geometry}
\geometry{top=1cm,bottom=1.2cm,left=1.25cm,right=1.25cm}
\setlength{\parindent}{0pt} \pagestyle{empty}

%------------------------------------------------------------------------
%   BIBLIOGRAPHY MANAGEMENT
%------------------------------------------------------------------------
\usepackage[
    backend=biber,
    style=authoryear,
    sorting=none, % manual order
    maxbibnames=99,
    doi=true,
    url=false
]{biblatex}
\addbibresource{publications.bib}

% Custom DOI format: No "DOI:" text, just a small clickable link
\DeclareFieldFormat{doi}{\href{https://doi.org/#1}{\footnotesize\nolinkurl{#1}}}

% Simple formatting to remove bibliography margins to match your layout
\defbibenvironment{bibliography}{%
    \list{}{%
        \setlength{\leftmargin}{0pt}%
        \setlength{\itemsep}{2pt}%
        \setlength{\parsep}{0pt}}} {\endlist} {%
    \item %
}

%------------------------------------------------------------------------
%   TABLES
%------------------------------------------------------------------------
\usepackage{array}
\usepackage{tabularx}
% Define a ragged-right fixed width column for dates to allow wrapping
\newcolumntype{L}[1]{>{\raggedright\let\newline\\\arraybackslash\hspace{0pt}}p{#1}}

%------------------------------------------------------------------------
%   GRAPHICS
%------------------------------------------------------------------------
\usepackage{graphicx}
\usepackage{transparent}
\usepackage{tikz}
\usetikzlibrary{shapes,backgrounds,mindmap,trees}

%------------------------------------------------------------------------
%   COLOR PALETTE (Blue Theme)
%------------------------------------------------------------------------
\usepackage{xcolor}
% Main blue (sidebar background, section headers)
\definecolor{sectcol}{RGB}{25,90,140}
% Accent blue (employer names, highlights)
\definecolor{accentcol}{RGB}{65,140,200}
% Light accent (stars, secondary highlights)
\definecolor{lightaccent}{RGB}{100,180,230}
% Dark gray (header, dates, secondary text)
\definecolor{bgcol}{RGB}{70,70,70}
% Light gray (divider lines)
\definecolor{softcol}{RGB}{220,220,220}

%------------------------------------------------------------------------
%   LINKS
%------------------------------------------------------------------------
\usepackage[colorlinks,urlcolor=accentcol]{hyperref}

%------------------------------------------------------------------------
%   FLEXIBLE 2-COLUMN LAYOUT
%------------------------------------------------------------------------
\usepackage{paracol}
\setlength{\columnsep}{8pt} % Slightly increased gap between main cols
\columnratio{0.75}

%------------------------------------------------------------------------
%   BREAKABLE COLORED SIDEBAR BOX
%------------------------------------------------------------------------
\usepackage[most]{tcolorbox}

%=======================================================================%
%   DEFINITIONS
%=======================================================================%
\newcommand{\mpwidth}{\linewidth-\fboxsep-\fboxsep}
\newcommand{\mystrut}{\rule[-.3\baselineskip]{0pt}{\baselineskip}}

% -- Arrows
\newcommand{\tzlarrow}{(0,0) -- (0.2,0) -- (0.3,0.2) -- (0.2,0.4) -- (0,0.4) -- (0.1,0.2) -- cycle;}
\newcommand{\larrow}[1]{\begin{tikzpicture}[scale=0.58]\filldraw[fill=#1,draw=#1!100!black] \tzlarrow\end{tikzpicture}}

% -- Section header
\newcommand{\cvsection}[1]{%
    \vspace{4pt}%
    \colorbox{sectcol}{\mystrut\makebox[1\mpwidth][l]{%
            \hspace{2pt}\larrow{bgcol}\hspace{-6pt}\larrow{bgcol}\hspace{-6pt}\larrow{bgcol}\hspace{6pt}%
            \textbf{\textcolor{white}{\uppercase{#1}}}%
        }}%
    \vspace{4pt}%
}

% -- Sidebar section title
\newenvironment{metasection}[1]{%
    \vspace{4pt}%
    \begin{center}%
        \textcolor{white}{\large{\uppercase{#1}}}\\
        \normalsize
        \parbox{0.7\mpwidth}{\textcolor{white}\hrule}
        \par\vspace{2pt} % <--- This forces a new line and adds a small gap
        }{%
    \end{center}%
}

% -- Icon helpers
\newcommand{\vcenteredhbox}[1]{\begingroup\setbox0=\hbox{#1}\parbox{\wd0}{\box0}\endgroup}

\newcommand{\icon}[3]{\makebox(#2,#2){\textcolor{#3}{\csname fa#1\endcsname}}}

\newcommand{\icontext}[4]{\vcenteredhbox{\icon{#1}{#2}{#4}}\
    \vcenteredhbox{\textcolor{#4}{#3}}}

\newcommand{\iconhref}[5]{\vcenteredhbox{\icon{#1}{#2}{#5}}\
    \href{#4}{\textcolor{#5}{#3}}}

\newcommand{\iconemail}[5]{\vcenteredhbox{\icon{#1}{#2}{#5}}\
    \href{mailto:#4}{\textcolor{#5}{#3}}}

% Custom icon command
\newcommand{\customicon}[2]{%
    \includegraphics[width=#2pt,height=#2pt]{#1}%
}
\newcommand{\customicontext}[4]{%
    \vcenteredhbox{\customicon{#1}{#2}}\ \vcenteredhbox{\textcolor{#4}{#3}}%
}

%=======================================================================%
%   EU-STYLE MAINENTRY
%=======================================================================%
\newcommand{\cvent}[5]{% 1=dates 2=position 3=employer 4=place 5=description
    \begin{tabularx}{1\mpwidth}{@{}L{0.13\mpwidth}@{\hspace{2ex}}X@{}}
        \textcolor{bgcol}{\footnotesize\textbf{#1}} &
        \textbf{#2} \vspace{1pt} \newline
        \textcolor{accentcol}{\textbf{#3}}\ifthenelse{\isempty{#4}}{}{, \textcolor{bgcol}{{\footnotesize\faMapMarker*}\ #4}} \vspace{1pt} \newline
        \small\textcolor{black}{#5}
    \end{tabularx}
    \vspace{4pt}\textcolor{softcol}{\hrule}\vspace{6pt}
}

%=======================================================================%
%   SIMPLE BULLET
%=======================================================================%
\newcommand{\cvbullet}[1]{%
    \parbox{1\mpwidth}{\larrow{softcol}\ #1}\vspace{3pt} }

%=======================================================================%
%   DOCUMENT
%=======================================================================%
\begin{document}

\begin{paracol}{2}

    %========================
    % MAIN COLUMN (left)
    %========================
    \colorbox{bgcol}{\makebox[\mpwidth][c]{%
            \HUGE{\textcolor{white}{\uppercase{Francesco Ioli}}}\ \textcolor{sectcol}{\rule[-1mm]{1mm}{0.9cm}}\ %
            \HUGE{\textcolor{white}{\uppercase{CV}}}%
        }}

    \vspace{6pt}
    \includegraphics[trim=250 150 0 40,clip,width=\linewidth]{assets/header.jpg}

    \transparent{0.75}\vspace{-120pt}\hspace{0.40\linewidth}
    \colorbox{bgcol}{\parbox{0.58\linewidth}{\transparent{1}
            \begin{center}
                \larrow{sectcol}\larrow{sectcol}
                \textcolor{white}{
                    I hold a PhD in Environmental and Infrastructure Engineering, specializing in Geoinformatics. My expertise lies in photogrammetry and 3D reconstruction, particularly in challenging environments such as alpine glaciers. I am proficient in coding and a passionate advocate for open-source software, data, and science. Currently, I am focused on deep learning image matching and am an experienced UAV pilot for photogrammetric and topographic surveys.
                }
            \end{center}}}
    \transparent{1}
    \vspace{18pt}

    % ---- Degree -----
    \cvsection{Degree}

    \cvent{2021 -- 2024}{PhD Environmental and Infrastructure Engineering}{Politecnico di Milano}{Milan, Italy}{%
        Major in Geoinformatics. \href{https://hdl.handle.net/10589/224092}{PhD Thesis}: \textit{Multi-temporal and Multi-scale photogrammetry for Alpine Glacier Monitoring}. Grade: Cum Laude.\newline
        I developed an image-based system with low-cost stereo cameras for short-term 4D glacier monitoring. I developed a software pipeline for daily 3D reconstruction with extreme-wide baseline between the stereo cameras using deep learning feature matching. I contributed to Deep-Image-Matching, a multi-view image matching library with deep learning for SfM. Additionally, I applied UAV photogrammetry for structural health assessment, including automated crack detection on concrete bridges, and cultural heritage documentation.
        %
    }

    % ----- Work experience -----
    \cvsection{Current employment}

    \cvent{06/2025 -- present}{Post-doctoral Researcher}{CNR IRPI}{Turin, Italy}{%
        Research on photogrammetry and 3D reconstruction pipelines for geo-hydrological hazard monitoring. Development of automated processing workflows for UAV and satellite imagery.%
    }

    \cvsection{Previous work experience}

    \cvent{07/2025 -- 12/2025}{External consultant (20\%) }{University of Zurich, Dept. of Geography}{Zurich, Switzerland}{%
        I completed the development of automated pipelines for regional-to-global-scale DEM reconstruction using satellite multi-view stereo, with applications in glacier mass balance studies. for Glambie-2 Glacier Mass Balance Intercomparison Exercise submission....Leveraged HPC clusters (Slurm) for large-scale multi-view stereo processing.
        %
    }

    \cvent{10/2024 -- 06/2025}{Post-doctoral Researcher}{University of Zurich, Dept. of Geography}{Zurich, Switzerland}{%
        I developed automated pipelines for large-scale DEM reconstruction using satellite multi-view stereo, with applications in glacier mass balance studies.%
    }

    \cvent{2022}{Topographic technical consultant}{Prof. Alberto Bianchi}{}{%
        Topographic consultant for the Technical Consultant of Office and Part (CTU) R.G. 717/2019%
    }

    \cvent{2022}{Topographic technician}{Gini Telecom}{}{%
        UAV surveys for telecommunication antennas%
    }

    % ---- Education and training -----
    \cvsection{Education and training}

    \cvent{04/2022-07/2022}{Visiting PhD student}{University of Twente, ITC (NL)}{}{%
        Development of a deep learning wide-baseline stereo matching workflow for 4D monitoring of an alpine glacier with low-cost time-lapse cameras.
        \href{https://doi.org/10.1007/s41064-023-00272-w}{[Paper]} \href{https://github.com/franioli/icepy4d}{[Code]}
    }

    \cvent{18 - 24/09/2022}{Summer School of Alpine Research}{University of Innsbruck}{Obergurgl (AT)}{%
        I participated in the Summer School \textit{Close Range Sensing Techniques in Alpine Terrain} organized by Innsbruck University
        with ISPRS support. \href{https://www.uibk.ac.at/iup/buecher/9783991060819.html}{[Proceedings]}%
    }

    \cvent{09/2019 - 02/2020}{Visiting student for MSc Thesis}{ETH Zürich, VAW (CH)}{}{%
        \textit{Evaluation of Airborne Image Velocimetry approaches with low-cost UAVs in riverine environments}. Supervisors: Prof. Livio Pinto{,} Dr. Martin
        Detert \href{http://doi.org/10.13140/RG.2.2.36679.65446}{[Thesis]} \href{https://doi.org/10.5194/isprs-archives-XLIII-B2-2020-597-2020}{[Paper]}%
    }

    \cvent{2020}{Internship}{Politecnico di Milano, Dept. of Civil and Environmental Engineering}{}{%   
        I learnt how to design and carry out topographic and UAV photogrammetric surveys for infrastructure and land monitoring. I learnt basics of AutoCAD for technical drawing from 3D point clouds. I obtained the A1/A3 and A2 UAV licenses with permission for flying in critical scenarios.
    }
    \cvent{2019}{Erasmus Exchange}{Aalto University}{Helsinki, Finland}{%
        Courses in remote sensing, GIS, and environmental engineering.%
    }

    \cvent{2017 -- 2020}{MSc Environmental Engineering}{Politecnico di Milano}{Milan, Italy}{%
        Major in Land Monitoring. Thesis on UAV photogrammetry for glacier monitoring. Grade: 110L/110.%
    }

    \cvent{2014 -- 2017}{BSc Environmental Engineering}{Politecnico di Milano}{Milan, Italy}{%
        Thesis on UAV snowpack surveys on Belvedere glacier. Grade: 102/110.%
    }

    %========================
    % SIDEBAR COLUMN (right)
    %========================
    \switchcolumn
    \begin{tcolorbox}[
            enhanced,
            breakable,
            colback=sectcol,
            colframe=sectcol,
            boxrule=0pt,
            arc=0pt,
            left=3mm,right=3mm,top=2mm,bottom=2mm
        ]

        \begin{metasection}{Personal Details}
            \icontext{MapMarker}{12}{Milan, Italy}{white}\\[4pt]
            \icontext{Male}{12}{03/09/1995}{white}\\[4pt]
            \iconhref{Orcid}{12}{\small 0000-0002-3184-7622}{https://orcid.org/0000-0002-3184-7622}{white}\\[4pt]
            \iconhref{Envelope}{12}{\small francesco.ioli@polimi.it}{francesco.ioli@polimi.it}{white}\\[4pt]
            \iconhref{Globe}{12}{franioli.github.io}{https://franioli.github.io}{white}\\[4pt]
            \iconhref{Github}{12}{github.com/franioli}{https://github.com/franioli}{white}\\[4pt]
            \iconhref{Linkedin}{12}{francesco-ioli}{https://www.linkedin.com/in/francesco-ioli-640061160/}{white}\\[4pt]
            \iconhref{GraduationCap}{12}{Google Scholar}{https://scholar.google.com/citations?user=hZkC2UMAAAAJ}{white}\\[4pt]
            \iconhref{ChartLine}{12}{H-Index: 12}{https://www.scopus.com/authid/detail.uri?authorId=57219022961}{white}\\[4pt]
        \end{metasection}

        \begin{metasection}{Languages}
            \textcolor{white}{
                \icontext{Comment}{10}{Italian}{white} (Native)\\[4pt]
                \icontext{GlobeAmericas}{10}{English}{white} C1 (IELTS 2019)
            }
        \end{metasection}

        \begin{metasection}{Programming}
            \textcolor{white}{
                \icontext{Python}{10}{Python}{white}\ \icon{Star}{8}{lightaccent}\icon{Star}{8}{lightaccent}\icon{Star}{8}{lightaccent}\icon{Star}{8}{lightaccent}\\[4pt]
                \icontext{Microchip}{10}{Matlab}{white}\ \icon{Star}{8}{lightaccent}\icon{Star}{8}{lightaccent}\icon{Star}{8}{lightaccent}\icon{Star}{8}{white}\\[4pt]
                \icontext{Code}{10}{C++}{white}\ \icon{Star}{8}{lightaccent}\icon{Star}{8}{white}\icon{Star}{8}{white}\icon{Star}{8}{white}\\[4pt]
                \icontext{Terminal}{10}{Bash/Shell}{white}\ \icon{Star}{8}{lightaccent}\icon{Star}{8}{lightaccent}\icon{Star}{8}{white}\icon{Star}{8}{white}\\[4pt]
            }
        \end{metasection}

        \begin{metasection}{Photogrammetry \& LiDAR}
            \textcolor{white}{
                \customicontext{assets/metashape}{10}{Agisoft Metashape}{white}\\[2pt]
                \customicontext{assets/photomodeler}{10}{Photomodeler}{white}\\[2pt]
                \customicontext{assets/cloudcompare}{10}{CloudCompare}{white}\\[2pt]
                \customicontext{assets/icon_pcd}{10}{COLMAP MicMac}{white}\\[2pt]
                \customicontext{assets/logo_odm}{10}{OpenMVG ODM}{white}\\[2pt]
            }
        \end{metasection}

        \begin{metasection}{Tools}
            \textcolor{white}{
                \customicontext{assets/pytorch}{10}{PyTorch}{white}\\[2pt]
                \icontext{Database}{10}{PostgreSQL PostGIS}{white}\\[2pt]
                \customicontext{assets/docker}{10}{Docker}{white}\\[2pt]
                \icontext{Cogs}{10}{Raspberry Arduino}{white}\\[2pt]
                \icontext{Github}{10}{Git GitHub}{white}\\[2pt]
                \icontext{FileCode}{10}{LaTeX}{white}\\[2pt]
            }
        \end{metasection}

        \begin{metasection}{Other Software}
            \textcolor{white}{
                \icontext{Globe}{10}{QGIS ESRI ArcGIS}{white}\\[2pt]
                \icontext{Satellite}{10}{RTKLib Leica Infinity}{white}\\[2pt]
                \icontext{Palette}{10}{Photoshop Lightroom}{white}\\[2pt]
                \icontext{FileImage}{10}{GIMP Inkscape}{white}\\[2pt]
                \icontext{DraftingCompass}{10}{AutoCAD}{white}\\[2pt]
            }
        \end{metasection}

        \begin{metasection}{Operating Systems}
            \textcolor{white}{
                \Large\icon{Linux}{20}{white}\ \Large\icon{Windows}{20}{white}
            }
        \end{metasection}

        \begin{metasection}{Infrastructure}
            \textcolor{white}{
                \icontext{Server}{10}{HPC}{white}\\[2pt]
                \footnotesize{Slurm workload manager}\\[6pt]
                \icontext{Cloud}{10}{Virtualization}{white}\\[2pt]
                \footnotesize{Proxmox \& OpenStack (client)}
            }
        \end{metasection}

        \begin{metasection}{Hobbies}
            \textcolor{white}{
                \customicon{assets/mountain}{20}\
                \icon{Camera}{18}{white}\
                \icon{Bicycle}{18}{white}
            }
        \end{metasection}

        \begin{metasection}{Certifications}
            \textcolor{white}{
                \icontext{Check}{10}{Professional Engineer}{white}\\[2pt]
                \footnotesize{Italian \textit{Esame di Stato} (Civil/Env)}\\[6pt]
                \icontext{Plane}{10}{UAS Pilot License}{white}\\[2pt]
                \footnotesize{EASA A2 Open \& Critical Scenario}\\[6pt]
                \icontext{Car}{10}{Driver's License (B)}{white}
            }
        \end{metasection}

    \end{tcolorbox}

\end{paracol}

%=======================================================================%
%   FULL WIDTH SECTIONS (Below the Sidebar)
%=======================================================================%
% These sections will now use 100% of the page width automatically.

% ---- Research funding, Grants -----
\cvsection{Research funding, Grants}

\cvbullet{Marie Curie Seal of Excellence -- MSCA Postdoctoral Fellowship 2024}

\cvbullet{Open Data Day 2024 Mini-Grant (Open Knowledge Foundation)}

\cvbullet{Best Paper Award -- ISPRS Geospatial Week 2023 (Youth Forum)}

% ----- Publications -----  
\cvsection{Research Outputs}

\begin{itemize}
    \item \textbf{Total number of publications:} xxxx (Source: Google Scholar / Scopus).
    \item \textbf{Metrics:} H-index: 12, Total Citations: xxxxx+ (as of Feb 2026).
    \item \textbf{Open Science:} 100\% of recent research outputs (2022--2025) are available via Open Access (DOI links provided below).
\end{itemize}

\footnotesize
% List your BibTeX keys here in the exact order you want them shown:
\nocite{
    gaspari2025glacier,
    bertulessi2025geomat,
    ioli2024pfg,
    morelli2024coregistering,
    morelli2024toolbox,
    gaspari2024heritage,
    gaspari2024teaching,
    morelli2024legacy,
    ioli2023multi,
    morelli2023slam,
    ioli2022rs,
    ioli2022bridge,
    gaspari2022integration,
    degaetani2021rs,
}
\vspace{-10pt} \printbibliography[heading=none] \vspace{-4pt}
\textit{For complete publication list, see \href{https://scholar.google.com/citations?user=hZkC2UMAAAAJ}{Google Scholar profile}}
\normalsize

% ----- Research supervision and leadership experience -----  
\cvsection{Research supervision and leadership experience}

\cvent{2024 -- present}{PhD Co-supervisor}{University of Zurich \& CNR IRPI}{}{%
    Co-supervision of PhD candidates ..... %
}

\cvent{2019 -- 2024}{MSc Thesis Co-supervisor}{Politecnico di Milano}{Milan, Italy}{%
    Supervised 6 Master's theses in Environmental Engineering:\newline
    {\footnotesize
        L. Cerina (2024): Satellite Stereo Images for Alpine Glacier Monitoring\newline
        S. Bonora (2024): Georeferenced Database for Belvedere Glacier\newline
        I. Pincolini (2022): DIC for Ice Flow Velocity Estimation\newline
        F. Barbieri (2021): Low-Cost UAV Photogrammetry for Alpine Areas\newline
        A. Pinto (2021): UAV Reconstruction of Bridge Fissures\newline
        F. Ferrario (2020): DGPS Assisted Aerial Triangulation for UAVs
    }
}

% ----- Teaching merits -----  
\cvsection{Teaching merits}

\cvent{2021 -- 2025}{Tutor in Summer Schools}{Politecnico di Milano}{Belvedere Glacier, Macugnaga, Italy}{%
    \textit{Design and Execution of Topographic Surveys for Land Monitoring} @ Belvedere Glacier aimed at introducing BSc and MSc students to topographic fieldwork in mountain environments. %
}

\cvent{2020 -- 2024}{Teaching Assistant}{Politecnico di Milano}{Milan, Italy}{%
    Provided academic support and laboratory tutoring for MSc and BSc courses: \newline
    {\footnotesize
        Photogrammetry and UAV surveying (MSc): Fall 2024\newline
        \textit{Trattamento delle Osservazioni} (Statistics) (BSc): Fall 2020, 2021, 2022, 2023\newline
        \textit{Sistemi Informativi Territoriali} (GIS) (BSc): Spring 2020, 2021\newline
        \textit{Tecniche di rilievo e modellazione 3D per l'architettura} (3D Modelling for Architecture) (BSc): Spring 2020, 2021, 2022.
    }
}

% ---- Awards and honours -----
\cvsection{Awards and honours}

\cvbullet{Winner of the prize for young researchers \textit{Premio Giovani 2023 – Sezione Ricerca} organized by the Italian Society of Photogrammetry and Topography SIFET during the congress \textit{65° Convegno Nazionale SIFET}, with the contribution \textit{Monitoraggio 4D ad alta frequenza di ghiacciai alpini tramite camere time-lapse a basso costo e Deep Learning Structure-from-Motion}.
}

\cvbullet{Finalist in the \href{https://blogs.egu.eu/geolog/2024/04/08/egu24-photo-competition-finalists-who-will-you-vote-for/}{EGU2024 Photo Competition}}

% ----- Additional information -----
% \cvsection{Additional information}

% \cvbullet{Languages: Italian (native), English (C1 -- IELTS 2019)}
% \cvbullet{UAS License: EASA A2 Open Category with Critical Scenario authorization}

% \cvbullet{Driver's license: B}

% \cvbullet{Professional qualification: Italian \textit{Esame di Stato} for civil and environmental engineers}

%=======================================================================%
%   FOOTER (Date and Consent)
%=======================================================================%
\vfill
\begin{center}
    \small\textit{According to EU Regulation 679/2016, I consent to the processing of my personal data.} \\
    \vspace{4pt}
    \textcolor{bgcol}{\hrulefill} \\
    \vspace{2pt}
    \textcolor{bgcol}{\footnotesize Francesco Ioli --- CV updated: \today}
\end{center}

\end{document}
