
\documentclass{cv_style} % <--- Load the custom CV class

% Add bibliography resource
\addbibresource{publications.bib}

%=======================================================================%
%   DOCUMENT CONTENT
%=======================================================================%
\begin{document}

\begin{paracol}{2}

    %========================
    % MAIN COLUMN (left)
    %========================
    \colorbox{bgcol}{\makebox[\mpwidth][c]{%
            \HUGE{\textcolor{white}{\uppercase{Francesco Ioli}}}\ \textcolor{sectcol}{\rule[-1mm]{1mm}{0.9cm}}\ %
            \HUGE{\textcolor{white}{\uppercase{CV}}}%
        }}

    \vspace{6pt}
    \includegraphics[trim=250 150 0 60,clip,width=\linewidth]{assets/header.jpg}

    \transparent{0.75}\vspace{-100pt}\hspace{0.40\linewidth}
    \colorbox{bgcol}{\parbox{0.58\linewidth}{\transparent{1}
            \begin{center}
                \larrow{sectcol}\larrow{sectcol}
                \textcolor{white}{
                    \textbf{Geomatics Post-doctoral Researcher} \\
                    Post-doc at CNR IRPI | PhD in Geoinformatics \\[4pt]
                    \footnotesize
                    Deep Learning Photogrammetry $\cdot$ 3D Reconstruction \\
                    Glacier monitoring $\cdot$ Open Science
                }
            \end{center}}}
    \transparent{1}
    \vspace{20pt}

    % ---- Profile -----
    Post-doctoral researcher at \textbf{CNR IRPI}, specializing in advanced photogrammetry and
    geoinformatics for environmental monitoring. My research bridges the gap between traditional
    geomatics and modern computer vision, focusing on \textbf{Deep Learning-based image matching} and
    automated 3D reconstruction pipelines.

    I possess extensive experience in multi-scale 4D monitoring, ranging from \textbf{low-cost
        terrestrial sensors} for high-frequency glacier dynamics to \textbf{satellite multi-view stereo}
    for regional mass balance studies (developed at \textbf{University of Zurich}). A strong advocate
    for Open Science, I actively develop open-source tools (e.g. Deep-Image-Matching) and manage
    large-scale geospatial datasets. My background includes rigorous field expertise as a certified UAV
    pilot and a strong commitment to teaching photogrammetry and statistics.

    % ---- Degree -----
    \cvsection{Degree}

    \cvent{2021 -- 2024}{PhD Environmental and Infrastructure Engineering}{Politecnico di Milano}{Milan, Italy}{%
        Major in Geoinformatics. \href{https://hdl.handle.net/10589/224092}{PhD Thesis}: \textit{Multi-temporal and Multi-scale photogrammetry for Alpine Glacier Monitoring}. Grade: Cum Laude.\newline
        I developed an image-based system with low-cost stereo cameras for short-term 4D glacier monitoring. I developed a software pipeline for daily 3D reconstruction with extreme-wide baseline between the stereo cameras using deep learning feature matching. I contributed to \href{https://github.com/3DOM-FBK/deep-image-matching}{Deep-Image-Matching}, a multi-view image matching library with deep learning for SfM. I applied UAV photogrammetry for structural health assessment, including automated crack detection on concrete bridges, and cultural heritage documentation.
        %
    }

    % ----- Work experience -----
    \cvsection{Current employment}

    \cvent{06/2025 -- present}{Post-doctoral Researcher}{CNR IRPI}{Turin, Italy}{%
        Research on photogrammetry and 3D reconstruction pipelines for geo-hydrological hazard monitoring. Development of automated processing workflows for UAV and satellite imagery.%
    }

    \cvsection{Previous work experience}

    \cvent{07/2025 -- 12/2025}{External consultant (20\%) }{University of Zurich, Dept. of Geography}{Zurich, Switzerland}{%
        I completed the development of automated pipelines for regional-to-global-scale DEM reconstruction using satellite multi-view stereo, with applications in glacier mass balance studies. for Glambie-2 Glacier Mass Balance Intercomparison Exercise submission....Leveraged HPC clusters (Slurm) for large-scale multi-view stereo processing.
        %
    }

    \cvent{10/2024 -- 06/2025}{Post-doctoral Researcher}{University of Zurich, Dept. of Geography}{Zurich, Switzerland}{%
        I developed automated pipelines for large-scale DEM reconstruction using satellite multi-view stereo, with applications in glacier mass balance studies.%
    }

    \cvent{2022}{Topographic technical consultant (part-time)}{Prof. Alberto Bianchi}{}{%
        Topographic consultant for the Technical Consultant of Office and Part (CTU) R.G. 717/2019%
    }

    \cvent{2022}{Topographic technician (part-time)}{Gini Telecom}{}{%
        UAV surveys for telecommunication antennas%
    }

    % ---- Education and training -----
    \cvsection{Education and training}

    \cvent{04/2022-07/2022}{Visiting PhD student}{University of Twente, ITC (NL)}{}{%
        Development of a deep learning wide-baseline stereo matching workflow for 4D monitoring of an alpine glacier with low-cost time-lapse cameras.
        \href{https://doi.org/10.1007/s41064-023-00272-w}{[Paper]} \href{https://github.com/franioli/icepy4d}{[Code]}
    }

    \cvent{18 - 24/\newline09/2022}{Summer School of Alpine Research}{University of Innsbruck}{Obergurgl (AT)}{%
        I participated in the Summer School \textit{Close Range Sensing Techniques in Alpine Terrain} organized by Innsbruck University
        with ISPRS support. \href{https://www.uibk.ac.at/iup/buecher/9783991060819.html}{[Proceedings]}%
    }

    \cvent{09/2019 - 02/2020}{Visiting student for MSc Thesis}{ETH Zürich, VAW (CH)}{}{%
        \textit{Evaluation of Airborne Image Velocimetry approaches with low-cost UAVs in riverine environments}. Supervisors: Prof. Livio Pinto{,} Dr. Martin
        Detert \href{http://doi.org/10.13140/RG.2.2.36679.65446}{[Thesis]} \href{https://doi.org/10.5194/isprs-archives-XLIII-B2-2020-597-2020}{[Paper]}%
    }

    \cvent{2020}{Internship}{Politecnico di Milano, Dept. of Civil and Environmental Engineering}{}{%   
        I learnt how to design and carry out topographic and UAV photogrammetric surveys for infrastructure and land monitoring. I learnt basics of AutoCAD for technical drawing from 3D point clouds. I obtained the A1/A3 and A2 UAV licenses with permission for flying in critical scenarios.
    }
    \cvent{2019}{Erasmus Exchange}{Aalto University}{Helsinki, Finland}{%
        Courses in remote sensing, GIS, and environmental engineering.%
    }

    \cvent{2017 -- 2020}{MSc Environmental Engineering}{Politecnico di Milano}{Milan, Italy}{%
        Major in Land Monitoring. Thesis on UAV photogrammetry for glacier monitoring. Grade: 110L/110.%
    }

    \cvent{2014 -- 2017}{BSc Environmental Engineering}{Politecnico di Milano}{Milan, Italy}{%
        Thesis on UAV snowpack surveys on Belvedere glacier. Grade: 102/110.%
    }

    %========================
    % SIDEBAR COLUMN (right)
    %========================
    \switchcolumn
    \begin{tcolorbox}[
            enhanced,
            breakable,
            colback=sectcol,
            colframe=sectcol,
            boxrule=0pt,
            arc=0pt,
            left=3mm,right=3mm,top=2mm,bottom=2mm
        ]

        \begin{metasection}{Personal Details}
            \icontext{MapMarker}{12}{Milan, Italy}{white}\\[4pt]
            \icontext{Male}{12}{03/09/1995}{white}\\[4pt]
            \iconhref{Orcid}{12}{\small 0000-0001-7429-891X}{https://orcid.org/0000-0001-7429-891X}{white}\\[4pt]
            \iconhref{Envelope}{12}{\small francescoioli@cnr.it}{francescoioli@cnr.it}{white}\\[4pt]
            % \iconhref{Globe}{12}{franioli.github.io}{https://franioli.github.io}{white}\\[4pt]
            \iconhref{Github}{12}{github.com/franioli}{https://github.com/franioli}{white}\\[4pt]
            \iconhref{Linkedin}{12}{francesco-ioli}{https://www.linkedin.com/in/francesco-ioli-640061160/}{white}\\[4pt]
            \iconhref{GraduationCap}{12}{Google Scholar}{https://scholar.google.com/citations?user=hZkC2UMAAAAJ}{white}\\[4pt]
            \iconhref{ChartLine}{12}{H-Index: 12}{https://www.scopus.com/authid/detail.uri?authorId=57219022961}{white}\\[4pt]
        \end{metasection}

        \begin{metasection}{Languages}
            \textcolor{white}{
                \icontext{Comment}{10}{Italian}{white} (Native)\\[4pt]
                \icontext{GlobeAmericas}{10}{English}{white} C1 (IELTS 2019)
            }
        \end{metasection}

        \begin{metasection}{Programming}
            \textcolor{white}{
                \icontext{Python}{10}{Python}{white}\ \icon{Star}{8}{lightaccent}\icon{Star}{8}{lightaccent}\icon{Star}{8}{lightaccent}\icon{Star}{8}{lightaccent}\\[4pt]
                \icontext{Microchip}{10}{Matlab}{white}\ \icon{Star}{8}{lightaccent}\icon{Star}{8}{lightaccent}\icon{Star}{8}{lightaccent}\icon{Star}{8}{white}\\[4pt]
                \icontext{Code}{10}{C++}{white}\ \icon{Star}{8}{lightaccent}\icon{Star}{8}{white}\icon{Star}{8}{white}\icon{Star}{8}{white}\\[4pt]
                \icontext{Terminal}{10}{Bash/Shell}{white}\ \icon{Star}{8}{lightaccent}\icon{Star}{8}{lightaccent}\icon{Star}{8}{white}\icon{Star}{8}{white}\\[4pt]
            }
        \end{metasection}

        \begin{metasection}{Photogrammetry \& LiDAR}
            \textcolor{white}{
                \customicontext{assets/metashape}{10}{Agisoft Metashape}{white}\\[2pt]
                \customicontext{assets/photomodeler}{10}{Photomodeler}{white}\\[2pt]
                \customicontext{assets/cloudcompare}{10}{CloudCompare}{white}\\[2pt]
                \customicontext{assets/icon_pcd}{10}{COLMAP MicMac}{white}\\[2pt]
                \customicontext{assets/logo_odm}{10}{OpenMVG ODM}{white}\\[2pt]
            }
        \end{metasection}

        \begin{metasection}{Tools}
            \textcolor{white}{
                \customicontext{assets/pytorch}{10}{PyTorch}{white}\\[2pt]
                \icontext{Database}{10}{PostgreSQL PostGIS}{white}\\[2pt]
                \customicontext{assets/docker}{10}{Docker}{white}\\[2pt]
                \icontext{Cogs}{10}{Raspberry Arduino}{white}\\[2pt]
                \icontext{Github}{10}{Git GitHub}{white}\\[2pt]
                \icontext{FileCode}{10}{LaTeX}{white}\\[2pt]
            }
        \end{metasection}

        \begin{metasection}{Other Software}
            \textcolor{white}{
                \icontext{Globe}{10}{QGIS ESRI ArcGIS}{white}\\[2pt]
                \icontext{Satellite}{10}{RTKLib Leica Infinity}{white}\\[2pt]
                \icontext{Palette}{10}{Photoshop Lightroom}{white}\\[2pt]
                \icontext{FileImage}{10}{GIMP Inkscape}{white}\\[2pt]
                \icontext{DraftingCompass}{10}{AutoCAD}{white}\\[2pt]
            }
        \end{metasection}

        \begin{metasection}{Operating Systems}
            \textcolor{white}{
                \Large\icon{Linux}{20}{white}\ \Large\icon{Windows}{20}{white}
            }
        \end{metasection}

        \begin{metasection}{Infrastructure}
            \textcolor{white}{
                \icontext{Server}{10}{HPC}{white}\\[2pt]
                \footnotesize{Slurm workload manager}\\[6pt]
                \icontext{Cloud}{10}{Virtualization}{white}\\[2pt]
                \footnotesize{Proxmox \& OpenStack (client)}
            }
        \end{metasection}

        \begin{metasection}{Hobbies}
            \textcolor{white}{
                \customicon{assets/mountain}{20}\
                \icon{Camera}{18}{white}\
                \icon{Bicycle}{18}{white}
            }
        \end{metasection}

        \begin{metasection}{Certifications}
            \textcolor{white}{
                \icontext{Check}{10}{Professional Engineer}{white}\\[2pt]
                \footnotesize{Italian \textit{Esame di Stato} (Civil/Env)}\\[6pt]
                \icontext{Plane}{10}{UAS Pilot License}{white}\\[2pt]
                \footnotesize{EASA A2 Open \& Critical Scenario}\\[6pt]
                \icontext{Car}{10}{Driver's License (B)}{white}
            }
        \end{metasection}

    \end{tcolorbox}

\end{paracol}

%=======================================================================%
%   FULL WIDTH SECTIONS (Below the Sidebar)
%=======================================================================%
% These sections will now use 100% of the page width automatically.

% ---- Research funding, Grants -----
\cvsection{Research funding and Grants}

\cvbullet{......MENTION TO CARIPLO PROJECT}

% ----- Publications -----  
\cvsection{Research Outputs}

\vspace{-6pt}
\begin{itemize}
    \setlength{\itemsep}{0pt}
    \setlength{\parskip}{0pt}
    \item \textbf{Total number of publications:} 24 (Source: Scopus).
    \item \textbf{Metrics:} H-index: 12, Total Citations: 272+ (as of Feb 2026).
    \item \textbf{Open Science:} 100\% of recent research outputs (2020--2025) are available via Open Access (DOI links provided below).
\end{itemize}

% --- SELECTED PUBLICATIONS SUB-BLOCK ---
\textbf{Selected Publications (10 most significant):} \vspace{-4pt}
\footnotesize
% List your BibTeX keys here in the exact order you want them shown:
\nocite{
    gaspari2025glacier,
    ioli2024pfg,
    morelli2024coregistering,
    morelli2024toolbox,
    gaspari2024heritage,
    morelli2023slam,
    ioli2022rs,
    ioli2022bridge,
    gaspari2022integration,
    degaetani2021rs%
}
\printbibliography[heading=none] \vspace{-4pt} \textit{For complete publication
    list, see \href{https://www.scopus.com/authid/detail.uri?authorId=57219022961}{Scopus profile: scopus.com/authid/detail.uri?authorId=57219022961}}
\normalsize%

% --- SOFTWARE, DATASETS & OTHER CONTRIBUTIONS ---
\vspace{4pt}
\textbf{Software, Datasets \& Infrastructure:} \vspace{-4pt}

\begin{itemize}
    \setlength{\itemsep}{2pt}
    \setlength{\parskip}{0pt}
    \item \textbf{Deep-Image-Matching} (Core Contributor): Toolbox for multi-view image matching with traditional and deep learning algorithms. \href{https://github.com/3DOM-FBK/deep-image-matching}{[GitHub]} \href{https://doi.org/10.5194/isprs-archives-XLVIII-2-W4-2024-309-2024}{[Paper]}

    \item \textbf{ICEpy4D} (Lead Developer): Open-source Python toolkit for 4D glacier monitoring using Deep Learning photogrammetry. \href{https://github.com/franioli/icepy4d}{[GitHub]} \href{https://doi.org/10.5194/isprs-archives-XLVIII-1-W2-2023-1037-2023}{[Paper]}

    \item \textbf{Belvedere Glacier Open Data \& Web-GIS Platform}: Curated dataset (Zenodo DOI: \href{https://doi.org/10.5281/zenodo.10817029}{10.5281/zenodo.10817029}) and interactive web platform for glacier documentation and storytelling. \href{https://thebelvedereglacier.it/}{[Web-app]}

    \item \textbf{Satellite Multi-View Stereo Pipeline}: Automated HPC workflow (Slurm) for regional-scale DEM reconstruction from satellite imagery, with applications in glacier mass balance (University of Zurich \& CNR IRPI).

    \item \textbf{COLMAP-SLAM Framework} (Core contributor): Visual odometry system for real-time photogrammetric positioning. \href{https://github.com/3DOM-FBK/COLMAP_SLAM}{[GitHub]} \href{https://doi.org/10.5194/isprs-archives-XLVIII-1-W1-2023-317-202}{[Paper]}
\end{itemize}

\vspace{4pt}
\textbf{PhD Thesis:} \vspace{-4pt}
\begin{itemize}
    \setlength{\itemsep}{2pt}
    \setlength{\parskip}{0pt}
    \item \textbf{PhD Thesis} (2024): \textit{Multi-temporal and Multi-scale photogrammetry for Alpine Glacier Monitoring}. Politecnico di Milano. Grade: Cum Laude. \href{https://hdl.handle.net/10589/224092}{[Handle]}
\end{itemize}

% ----- Research supervision and leadership experience -----  
\cvsection{Research supervision and leadership experience}

\cventwide{2024 -- present}{PhD Co-supervisor}{University of Zurich \& CNR IRPI}{}{%
    Co-supervision of PhD candidates ..... %
}

\cventwide{2019 -- 2024}{MSc Thesis Co-supervisor}{Politecnico di Milano}{Milan, Italy}{%
    Supervised 6 Master's theses in Environmental and Land Planning Engineering:\newline
    \minibull L. Cerina (2024): Very-High Resolution Satellite Stereo Images for Alpine
    Glacier Monitoring. Supervisor: prof.\ L. Pinto.\newline
    \minibull S. Bonora (2024): Progettazione e implementazione di un database georeferenziato per il monitoraggio del Ghiacciaio del Belvedere. Supervisor: prof.\ F. Migliaccio.\newline
    \minibull I. Pincolini (2022): Digital Image Correlation for ice flow velocity estimation: a case study on the Belvedere Glacier with UAV orthophotos. Supervisor: prof.\ L. Pinto.\newline
    \minibull F. Barbieri (2021): Monitoraggio di aree alpine inaccessibili con fotogrammetria UAV low-cost. Supervisor: prof.\ L. Pinto.\newline
    \minibull A. Pinto (2021): Tecniche fotogrammetriche da drone per la ricostruzione metrica di fessure su ponti in calcestruzzo. Supervisor: prof.\ L. Pinto.\newline
    \minibull F. Ferrario (2020): Triangolazione aerea assistita da DGPS in fotogrammetria da UAV: sperimentazione di una soluzione a basso costo per il DJI Matrice 210 V2. Supervisor: prof.\ L. Pinto.\newline
}

% ----- Teaching merits -----  
\cvsection{Teaching merits}

\cventwide{2020 -- 2024}{Teaching Assistant}{Politecnico di Milano}{Milan, Italy}{%
    Provided academic support and laboratory tutoring for MSc and BSc courses: \newline
    {\footnotesize
        Photogrammetry and UAV surveying (MSc): Fall 2024\newline
        \textit{Trattamento delle Osservazioni} (Statistics) (BSc): Fall 2020, 2021, 2022, 2023\newline
        \textit{Sistemi Informativi Territoriali} (GIS) (BSc): Spring 2020, 2021\newline
        \textit{Tecniche di rilievo e modellazione 3D per l'architettura} (3D Modelling for Architecture) (BSc): Spring 2020, 2021, 2022.
    }
}

\cventwide{2021 -- 2025}{Tutor in Summer Schools}{Politecnico di Milano}{Belvedere Glacier, Macugnaga, Italy}{%
    \textit{Design and Execution of Topographic Surveys for Land Monitoring} @ Belvedere Glacier aimed at introducing BSc and MSc students to topographic fieldwork in mountain environments. %
}

\cventwide{2024}{Open Data Day 2024}{Open Knowledge Foundation}{}{%
    Awarded for the \href{https://blog.okfn.org/2024/02/28/and-the-winners-of-the-open-data-day-2024-mini-grants-are/}{ Open Data Day 2024 mini-grant} for the organization of the webinar \href{https://labmgf.dica.polimi.it/opendataday/}{\textit{Mapping Climate Change in 4D: Belvedere Glacier’s Open Geo Data for Education and Research}} \href{https://blog.okfn.org/2024/03/28/oddstories-2024-milan-italy/}{[Event Report]}
}

\cventwide{2023}{EGU Higher Education Teaching Grant 2023}{EGU}{}{%
    Winner of the \href{https://www.egu.eu/education/teg/hetg/2023/}{EGU Higher Education Teaching Grant 2023} for the open teaching material for the Summer School "Design and implementation of topographic surveys for territorial monitoring in mountain environments" \href{https://tars4815.github.io/belvedere-summer-school/}{[Teaching material]}
}

% ---- Awards and honours -----
\cvsection{Awards and honours}

\cvbullet{Marie Curie Seal of Excellence -- MSCA Postdoctoral Fellowship 2024 (score: 92.8) and 2025 (score: 95.4)}

\cvbullet{Winner of the prize for young researchers \textit{Premio Giovani 2023 – Sezione Ricerca} organized by the Italian Society of Photogrammetry and Topography SIFET during the congress \textit{65° Convegno Nazionale SIFET}, with the contribution \textit{Monitoraggio 4D ad alta frequenza di ghiacciai alpini tramite camere time-lapse a basso costo e Deep Learning Structure-from-Motion}.
}

\cvbullet{Finalist in the \href{https://blogs.egu.eu/geolog/2024/04/08/egu24-photo-competition-finalists-who-will-you-vote-for/}{EGU2024 Photo Competition}}

\cvsection{Other Key Academic Merits}

% --- Conference Presentations ---
\vspace{4pt}
\textbf{\footnotesize Presentations in Scientific Conferences}
\small
\begin{itemize}
    \setlength{\itemsep}{1pt}
    \setlength{\parskip}{0pt}

    \item \textbf{2025:} EGU General Assembly, Vienna (Oral) \href{https://meetingorganizer.copernicus.org/EGU25/EGU25-15978.html}{[Abstract]}; Alpine Glaciology Meeting, Innsbruck (Poster).

    \item \textbf{2024:} EGU General Assembly, Vienna (Oral) \href{https://meetingorganizer.copernicus.org/EGU24/EGU24-16412.html}{[Abstract]}.

    \item \textbf{2023:} ISPRS Geospatial Week, Cairo (Oral); EGU General Assembly, Vienna (Oral); VGC, Dresden (Oral); SIFET Congress, Arezzo (Oral); GeoAI, Turin (Oral).

    \item \textbf{2022:} EGU General Assembly, Vienna (Oral); ISPRS Congress, Nice (Poster).
\end{itemize}

% --- Peer Review / Memberships (If you have them) ---
\vspace{6pt}
\textbf{\footnotesize Memberships \& Peer Review}
\begin{itemize}
    \item Reviewer for: \textit{ISPRS Journal of Photogrammetry and Remote Sensing}, \textit{Remote Sensing},
          \textit{The Cryosphere}.
    \item Member of: EGU (European Geosciences Union), SIFET (Italian Society of Photogrammetry and
          Topography).
\end{itemize}
\normalsize

% ----- Additional information -----
% \cvsection{Additional information}

% \cvbullet{Languages: Italian (native), English (C1 -- IELTS 2019)}
% \cvbullet{UAS License: EASA A2 Open Category with Critical Scenario authorization}

% \cvbullet{Driver's license: B}

% \cvbullet{Professional qualification: Italian \textit{Esame di Stato} for civil and environmental engineers}

%=======================================================================%
%   FOOTER (Date and Consent)
%=======================================================================%
\vfill
\begin{center}
    \small\textit{According to EU Regulation 679/2016, I consent to the processing of my personal data.} \\
    \vspace{4pt}
    \textcolor{bgcol}{\hrulefill} \\
    \vspace{2pt}
    \textcolor{bgcol}{\footnotesize Francesco Ioli --- CV updated: \today}
\end{center}

\end{document}
