\documentclass{cv_style} % <--- Load the custom CV class

% Add bibliography resource
\addbibresource{publications.bib}

%=======================================================================%
%   DOCUMENT CONTENT
%=======================================================================%
\begin{document}

\begin{paracol}{2}

    %========================
    % LEFT COLUMN (PHOTO)
    %========================

    % 1. Hard reset + Strut
    \vspace*{-\topskip} % Cancels default top glue
    \vspace*{\headerstrut} % Adds fixed space

    % 2. Image
    % Use clip to trim if needed, but display vertically
    \includegraphics[width=\linewidth, clip]{assets/P_UZHGEO_FioliFrancesco_250306_0120_WEB.jpg}

    %========================
    % RIGHT COLUMN (HEADER + PROFILE)
    %========================
    \switchcolumn % <--- Switch to Right Column

    % 1. Hard reset + Strut (MUST MATCH EXACTLY)
    \vspace*{-\topskip}
    \vspace*{\headerstrut}

    % Name Header
    {\noindent\color{sectcol}\HUGE\textbf{\uppercase{Francesco Ioli}}, \Large{\textbf{PhD}}} \\[6pt]
    {\Large\textbf{Geomatics Researcher at CNR IRPI}} \\[4pt]
    {\color{bgcol}\rule{\linewidth}{1pt}} % Horizontal divider
    \vspace{0em} % Space after divider

    % ---- Profile -----
    Post-doctoral researcher at the \textbf{Research Institute for Geo-Hydrological Protection IRPI} at
    CNR (Italy), specialising in \textbf{photogrammetry and geoinformatics for environmental
        monitoring}. My research bridges the gap between traditional photogrammetry and modern computer
    vision, focusing on deep learning-based image matching and automated 3D reconstruction pipelines.

    I have experience in multi-scale 4D monitoring, ranging from \textbf{low-cost terrestrial sensors}
    for high-frequency glacier dynamics to \textbf{satellite multi-view stereo} for regional mass
    balance studies (developed at University of Zurich).

    I have strong coding skills and experience with HPC for processing large-scale geospatial data. As
    a strong advocate for open science, I actively develop open-source tools. My background includes
    field expertise as a certified UAV pilot and topographer.

\end{paracol} % Close the 2-column layout here temporarily

\vspace{8pt}

% Set background for the left column (index 0)
\backgroundcolor{c[0]}{sectcol}

\begin{paracol}{2}
    %========================
    % LEFT COLUMN (SIDEBAR)
    %========================

    % Force Reset
    \vspace*{-\topskip}
    \vspace*{0pt}

    % Sidebar Content
    % \begin{tcolorbox}[
    %         enhanced,
    %         breakable,
    %         colback=sectcol,
    %         colframe=sectcol,
    %         boxrule=0pt,
    %         arc=0pt,
    %         left=3mm,right=3mm,top=2mm,bottom=2mm
    %     ]

    \begin{metasection}{Personal Details}
        \icontext{MapMarker}{12}{Turin, Italy}{white}\\[4pt]
        \icontext{Male}{12}{03/09/1995}{white}\\[4pt]
        \iconhref{Orcid}{12}{\small 0000-0001-7429-891X}{https://orcid.org/0000-0001-7429-891X}{white}\\[4pt]
        \iconhref{Envelope}{12}{\small francescoioli@cnr.it}{francescoioli@cnr.it}{white}\\[4pt]
        % \iconhref{Globe}{12}{franioli.github.io}{https://franioli.github.io}{white}\\[4pt]
        \iconhref{Github}{12}{github.com/franioli}{https://github.com/franioli}{white}\\[4pt]
        \iconhref{Linkedin}{12}{francesco-ioli}{https://www.linkedin.com/in/francesco-ioli-640061160/}{white}\\[4pt]
        \iconhref{GraduationCap}{12}{Google Scholar}{https://scholar.google.com/citations?user=hZkC2UMAAAAJ}{white}\\[4pt]
        \iconhref{ChartLine}{12}{H-Index: 12}{https://www.scopus.com/authid/detail.uri?authorId=57219022961}{white}\\[4pt]
    \end{metasection}

    \begin{metasection}{Languages}
        \textcolor{white}{
            \icontext{Comment}{10}{Italian}{white} (Native)\\[4pt]
            \icontext{GlobeAmericas}{10}{English}{white} C1 (IELTS 2019)
        }
    \end{metasection}

    \begin{metasection}{Programming}
        \textcolor{white}{
            \icontext{Python}{10}{Python}{white}\ \icon{Star}{8}{lightaccent}\icon{Star}{8}{lightaccent}\icon{Star}{8}{lightaccent}\icon{Star}{8}{lightaccent}\\[4pt]
            \icontext{Microchip}{10}{Matlab}{white}\ \icon{Star}{8}{lightaccent}\icon{Star}{8}{lightaccent}\icon{Star}{8}{lightaccent}\icon{Star}{8}{white}\\[4pt]
            \icontext{Code}{10}{C++}{white}\ \icon{Star}{8}{lightaccent}\icon{Star}{8}{white}\icon{Star}{8}{white}\icon{Star}{8}{white}\\[4pt]
            \icontext{Terminal}{10}{Bash/Shell}{white}\ \icon{Star}{8}{lightaccent}\icon{Star}{8}{lightaccent}\icon{Star}{8}{lightaccent}\icon{Star}{8}{white}\\[4pt]
        }
    \end{metasection}

    \begin{metasection}{Photogrammetry \& LiDAR}
        \textcolor{white}{
            \customicontext{assets/metashape}{10}{Agisoft Metashape}{white}\\[2pt]
            \customicontext{assets/photomodeler}{10}{Photomodeler}{white}\\[2pt]
            \customicontext{assets/cloudcompare}{10}{CloudCompare}{white}\\[2pt]
            \customicontext{assets/icon_pcd}{10}{COLMAP $\cdot$ MicMac}{white}\\[2pt]
            \customicontext{assets/logo_odm}{10}{OpenMVG $\cdot$ OpenDroneMap}{white}\\[2pt]
        }
    \end{metasection}

    \newpage % Force new page for better layout
    \vspace*{-\topskip}
    \vspace*{0pt} % Force top margin on the new page

    \begin{metasection}{Tools}
        \textcolor{white}{
            \customicontext{assets/pytorch}{10}{PyTorch}{white}\\[2pt]
            \icontext{Database}{10}{PostgreSQL $\cdot$ PostGIS}{white}\\[2pt]
            \customicontext{assets/docker}{10}{Docker}{white}\\[2pt]
            \icontext{Cogs}{10}{Raspberry $\cdot$ Arduino}{white}\\[2pt]
            \icontext{Github}{10}{Git $\cdot$ GitHub}{white}\\[2pt]
            \icontext{FileCode}{10}{LaTeX}{white}\\[2pt]
        }
    \end{metasection}

    \begin{metasection}{Other Software}
        \textcolor{white}{
            \icontext{Globe}{10}{QGIS $\cdot$ ESRI ArcGIS}{white}\\[2pt]
            \icontext{Satellite}{10}{RTKLib $\cdot$ Leica Infinity}{white}\\[2pt]
            \icontext{Palette}{10}{Photoshop $\cdot$ Lightroom}{white}\\[2pt]
            \icontext{FileImage}{10}{GIMP $\cdot$ Inkscape}{white}\\[2pt]
            \icontext{DraftingCompass}{10}{AutoCAD}{white}\\[2pt]
        }
    \end{metasection}

    \begin{metasection}{Operating Systems}
        \icontext{Linux}{10}{Linux}{white}\\[2pt]
        \icontext{Windows}{10}{Windows}{white}\\[2pt]
    \end{metasection}

    \begin{metasection}{HPC Infrastructure}
        \textcolor{white}{
            \icontext{Server}{10}{Slurm workload manager}{white}\\[4pt]
            \icontext{Microchip}{10}{Virtualization}{white}\\[2pt]
            \footnotesize{Proxmox $\cdot$ OpenStack (client)}\\[4pt]
            \icontext{Cloud}{10}{Cloud Computing}{white}\\[2pt]
            \footnotesize{DigitalOcean $\cdot$ Lambda.ai}
        }
    \end{metasection}

    \begin{metasection}{Hobbies}
        \textcolor{white}{
            \customicon{assets/mountain}{20}\
            \icon{Camera}{18}{white}\
            \icon{Bicycle}{18}{white}
        }
    \end{metasection}

    % \begin{metasection}{Certifications}
    %     \textcolor{white}{
    %         \icontext{Check}{10}{Professional Engineer}{white}\\[2pt]
    %         \footnotesize{Italian \textit{Esame di Stato} (Civil/Env)}\\[6pt]
    %         \icontext{Plane}{10}{UAS Pilot License}{white}\\[2pt]
    %         \footnotesize{EASA A2 Open \& Critical Scenario}\\[6pt]
    %         \icontext{Car}{10}{Driver's License (B)}{white}
    %     }
    % \end{metasection}

    \switchcolumn % Switch to Right column

    %========================
    % RIGHT COLUMN (MAIN)
    %========================
    % Force Reset
    \vspace*{-\topskip}
    \vspace*{0pt}

    % ---- Degree -----
    \cvsection{Degree}

    \cvent[noline]{2021 -- 2024}{PhD Environmental and Infrastructure Engineering}{Politecnico di Milano}{Milan, Italy}{%
        Major in Geomatics. Grade: Cum Laude.\newline
        Thesis: \href{https://hdl.handle.net/10589/224092}{\textit{Multi-temporal and Multi-scale photogrammetry for Alpine Glacier Monitoring}}. Supervisor: prof. L. Pinto, co-supervisor: prof. F. Nex. \newline
        I developed an image-based system with low-cost stereo cameras for short-term 4D glacier monitoring. I developed a software pipeline for daily 3D reconstruction with extreme-wide baseline between the stereo cameras using deep learning feature matching. I contributed to \href{https://github.com/3DOM-FBK/deep-image-matching}{Deep-Image-Matching}, a multi-view image matching library with deep learning for SfM. I applied UAV photogrammetry for structural health assessment, including automated crack detection on concrete bridges, and cultural heritage documentation.
        %
    }

    % ----- Work experience -----
    \cvsection{Current employment}

    \cvent[noline]{06/2025 -- present}{Post-doctoral Researcher}{CNR IRPI}{Turin, Italy}{%
        Developing automated low-cost image analysis workflows for glacier instability risk management at Planpincieux Glacier (Mont Blanc, Italy). Enhancing the current monitoring systems using terrestrial stereoscopic time-lapse cameras for kinematics detection, ice avalanche monitoring, and velocity computation via Digital Image Correlation and Structure-from-Motion techniques. Results directly support expert decision-making for glacier instability risk mitigation.
        Working on historical aerial photogrammetry for the reconstruction of periglacier debris volume estimation.
    }

    % ---- Previous work experience -----
    \cvsection{Previous work experience}

    \cvent{07/2025 -- 12/2025}{External consultant (20\%) }{University of Zurich, Dept. of Geography}{Zurich, Switzerland}{%
        Finalized automated pipelines for regional-to-global scale Digital Elevation Model (DEM) reconstruction from satellite multi-view stereo for the Glambie-2 Glacier Mass Balance Intercomparison Exercise using UZH Slurm-based HPC clusters.%
    }

    \cvent{10/2024 -- 06/2025}{Post-doctoral Researcher}{University of Zurich, Dept. of Geography}{Zurich, Switzerland}{%
        Developed automated pipelines for large-scale DEM reconstruction using satellite multi-view stereo for regional glacier mass balance assessments.
    }

    \cvent{2022}{Topographic technical consultant (part-time)}{Prof. Alberto Bianchi}{}{%
        Topographic consultant for the Technical Consultant of Office and Part (CTU) R.G. 717/2019%
    }

    \cvent[noline]{2022}{Topographic technician (part-time)}{Gini Telecom}{}{%
        UAV surveys for telecommunication antennas%
    }

    % ---- Education and training -----
    \cvsection{Education and training}

    \cvent{04/2022 -- 07/2022}{Visiting PhD student}{University of Twente, ITC}{Enschede (NL)}{%
        Developed a deep learning-based wide-baseline stereo matching workflow for 4D alpine glacier monitoring with low-cost time-lapse cameras.
        \href{https://doi.org/10.1007/s41064-023-00272-w}{[Paper]} \href{https://github.com/franioli/icepy4d}{[Code]}
    }

    \cvent{18 -- 24/\newline09/2022}{Summer School of Alpine Research}{University of Innsbruck}{Obergurgl (AT)}{%
        Summer School \textit{Close Range Sensing Techniques in Alpine Terrain} organized by Innsbruck University
        with ISPRS support. \href{https://www.uibk.ac.at/iup/buecher/9783991060819.html}{[Proceedings]}%
    }

    \cvent{09/2019 -- 02/2020}{Visiting student for MSc Thesis}{Dept. VAW, ETH Zürich}{Zürich (CH)}{%
        MSc Thesis: \href{http://doi.org/10.13140/RG.2.2.36679.65446}{\textit{Evaluation of Airborne Image Velocimetry approaches with low-cost UAVs in riverine environments}}.
        Supervisors: Prof. Livio Pinto{,} Dr. Martin Detert.
        \href{https://doi.org/10.5194/isprs-archives-XLIII-B2-2020-597-2020}{[Paper]}%
    }

    \cvent{2020}{Internship}{Dept. of Civil and Environmental Engineering, Politecnico di Milano}{}{%   
        Performed topographic and UAV surveys for infrastructure health monitoring and territorial monitoring. Learnt basics of CAD modeling for technical drawing from 3D point clouds.
        Obtained A1/A3 and A2 UAV licenses with permission for flying in critial standard scenarios.
    }
    \cvent{2019}{Erasmus Exchange}{Aalto University}{Helsinki, Finland}{%
        Exchange semester with courses in laser scanning, remote sensing, image processing, hydrological modelling.%
    }

    \cvent{2017 -- 2020}{MSc Environmental and Land Planning Engineering}{Politecnico di Milano}{Milan, Italy}{%
        Major in Land Monitoring and Diagnostics. Grade: 110L/110.%
    }

    \cvent[noline]{2014 -- 2017}{BSc Environmental and Land Planning Engineering}{Politecnico di Milano}{Milan, Italy}{%
        Grade: 102/110.%
    }

\end{paracol}

\vspace{8pt}

%=======================================================================%
%   FULL WIDTH SECTIONS (Below the Sidebar)
%=======================================================================%
% These sections will now use 100% of the page width automatically.

% ---- Research funding, Grants -----
\cvsection{Research funding and Grants}

\cvbullet{
    \textbf{MOHYCAM} –- \textit{Modified HYdrogeological hazards under complex ClimAte and environmental conditions: Monitoring activities and mitigation strategies} [2025--2027] (Fondazione Cariplo, Grant \#2024-3388 \href{https://www.fondazionecariplo.it/progetto/territori-sicuri/}{\textit{Territori Sicuri}}, €677,735; PI: Prof. Livio Pinto, Politecnico di Milano).\newline
    \textbf{Contributed to proposal writing} for addressing hydrogeological vulnerabilities at the Belvedere Glacier (Monte Rosa), by  combining advanced glacier dynamics monitoring with community-based risk awareness strategies.
}

% ----- Publications -----  
\cvsection{Research Outputs}

\vspace{-6pt}
\begin{itemize}
    \setlength{\itemsep}{0pt}
    \setlength{\parskip}{0pt}
    \item \textbf{Total number of publications:} 24 (Source: Scopus).
    \item \textbf{Metrics:} H-index: 12, Total Citations: 272+ (as of Feb 2026).
    \item \textbf{Open Science:} 100\% of recent research outputs (2020--2025) are available via Open Access (DOI links provided below).
\end{itemize}

% --- SELECTED PUBLICATIONS SUB-BLOCK ---
\textbf{Selected Publications (10 most significant):} \vspace{-4pt}
\footnotesize
% List your BibTeX keys here in the exact order you want them shown:
\nocite{
    gaspari2025glacier,
    ioli2024pfg,
    morelli2024coregistering,
    morelli2024toolbox,
    gaspari2024heritage,
    morelli2023slam,
    ioli2022rs,
    ioli2022bridge,
    gaspari2022integration,
    degaetani2021rs%
}
\printbibliography[heading=none] \vspace{-4pt} \textit{For complete publication
    list, see \href{https://www.scopus.com/authid/detail.uri?authorId=57219022961}{Scopus profile: scopus.com/authid/detail.uri?authorId=57219022961}}
\normalsize%

% --- SOFTWARE, DATASETS & OTHER CONTRIBUTIONS ---
\vspace{4pt}
\textbf{Software, Datasets \& Infrastructure:} \vspace{-4pt}

\begin{itemize}
    \setlength{\itemsep}{2pt}
    \setlength{\parskip}{0pt}
    \item \textbf{Deep-Image-Matching} (Core Contributor): Toolbox for multi-view image matching with traditional and deep learning algorithms. \href{https://github.com/3DOM-FBK/deep-image-matching}{[GitHub]} \href{https://doi.org/10.5194/isprs-archives-XLVIII-2-W4-2024-309-2024}{[Paper]}

    \item \textbf{ICEpy4D} (Lead Developer): Open-source Python toolkit for 4D glacier monitoring using Deep Learning photogrammetry. \href{https://github.com/franioli/icepy4d}{[GitHub]} \href{https://doi.org/10.5194/isprs-archives-XLVIII-1-W2-2023-1037-2023}{[Paper]}

    \item \textbf{Belvedere Glacier Open Data \& Web-GIS Platform}: Curated dataset (Zenodo DOI: \href{https://doi.org/10.5281/zenodo.10817029}{10.5281/zenodo.10817029}) and interactive web platform for glacier documentation and storytelling. \href{https://thebelvedereglacier.it/}{[Web-app]}

    \item \textbf{Satellite Multi-View Stereo Pipeline}: Automated HPC workflow (Slurm) for regional-scale DEM reconstruction from satellite imagery, with applications in glacier mass balance (University of Zurich).

    \item \textbf{COLMAP-SLAM Framework} (Core contributor): Visual odometry system for real-time photogrammetric positioning. \href{https://github.com/3DOM-FBK/COLMAP_SLAM}{[GitHub]} \href{https://doi.org/10.5194/isprs-archives-XLVIII-1-W1-2023-317-202}{[Paper]}
\end{itemize}

\vspace{4pt}
\textbf{PhD Thesis:} \vspace{-4pt}
\begin{itemize}
    \setlength{\itemsep}{2pt}
    \setlength{\parskip}{0pt}
    \item \textbf{PhD Thesis} (2024): \textit{Multi-temporal and Multi-scale photogrammetry for Alpine Glacier Monitoring}. Politecnico di Milano. Grade: Cum Laude. \href{https://hdl.handle.net/10589/224092}{[Handle]}
\end{itemize}

% ----- Research supervision and leadership experience -----  
\cvsection{Research supervision and leadership experience}

\cventwide{2024 -- present}{PhD Student Mentoring}{University of Zurich \& CNR IRPI}{}{%
    Mentoring and co-supervision of PhD candidates R. Pedrelli (UZH) and D. Cardone (CNR IRPI).
    Providing technical guidance on photogrammetric pipelines, satellite multi-view stereo, and glacier monitoring workflows.
}

\cventwide{2019 -- 2024}{MSc Thesis Co-supervisor}{Politecnico di Milano}{Milan, Italy}{%
    Supervised 6 Master's theses in Environmental and Land Planning Engineering:
    \begin{itemize}[nosep]
        \item L. Cerina (2024): Very-High Resolution Satellite Stereo Images for Alpine Glacier Monitoring.
              Supervisor: prof.\ L. Pinto.
        \item S. Bonora (2024): Progettazione e implementazione di un database georeferenziato per il
              monitoraggio del Ghiacciaio del Belvedere. Supervisor: prof.\ F. Migliaccio.
        \item I. Pincolini (2022): Digital Image Correlation for ice flow velocity estimation: a case study on
              the Belvedere Glacier with UAV orthophotos. Supervisor: prof.\ L. Pinto.
        \item F. Barbieri (2021): Monitoraggio di aree alpine inaccessibili con fotogrammetria UAV low-cost.
              Supervisor: prof.\ L. Pinto.
        \item A. Pinto (2021): Tecniche fotogrammetriche da drone per la ricostruzione metrica di fessure su
              ponti in calcestruzzo. Supervisor: prof.\ L. Pinto.
        \item F. Ferrario (2020): Triangolazione aerea assistita da DGPS in fotogrammetria da UAV:
              sperimentazione di una soluzione a basso costo per il DJI Matrice 210 V2. Supervisor: prof.\ L.
              Pinto.
    \end{itemize}
}

% ----- Teaching merits -----  
\cvsection{Teaching merits}

\cventwide{2020 -- 2024}{Teaching Assistant}{Politecnico di Milano}{Milan, Italy}{%
    Provided academic support and laboratory tutoring for MSc and BSc courses: \newline
    {\footnotesize
        Photogrammetry and UAV surveying (MSc): Fall 2024\newline
        \textit{Trattamento delle Osservazioni} (Statistics) (BSc): Fall 2020, 2021, 2022, 2023\newline
        \textit{Sistemi Informativi Territoriali} (GIS) (BSc): Spring 2020, 2021\newline
        \textit{Tecniche di rilievo e modellazione 3D per l'architettura} (3D Modelling for Architecture) (BSc): Spring 2020, 2021, 2022.
    }
}

\cventwide{2021 -- 2025}{Tutor in Summer Schools}{Politecnico di Milano}{Belvedere Glacier, Macugnaga, Italy}{%
    \textit{Design and Execution of Topographic Surveys for Land Monitoring} @ Belvedere Glacier aimed at introducing BSc and MSc students to topographic fieldwork in mountain environments. %
}

\cventwide{2024}{Open Data Day 2024}{Open Knowledge Foundation}{}{%
    Awarded for the \href{https://blog.okfn.org/2024/02/28/and-the-winners-of-the-open-data-day-2024-mini-grants-are/}{ Open Data Day 2024 mini-grant} for the organization of the webinar \href{https://labmgf.dica.polimi.it/opendataday/}{\textit{Mapping Climate Change in 4D: Belvedere Glacier’s Open Geo Data for Education and Research}} \href{https://blog.okfn.org/2024/03/28/oddstories-2024-milan-italy/}{[Event Report]}
}

\cventwide{2023}{EGU Higher Education Teaching Grant 2023}{EGU}{}{%
    Winner of the \href{https://www.egu.eu/education/teg/hetg/2023/}{EGU Higher Education Teaching Grant 2023} for the open teaching material for the Summer School "Design and implementation of topographic surveys for territorial monitoring in mountain environments" \href{https://tars4815.github.io/belvedere-summer-school/}{[Teaching material]}
}

% ---- Awards and honours -----
\cvsection{Awards and honours}

\cvbullet{Marie Curie Seal of Excellence -- MSCA Postdoctoral Fellowship 2024 (score: 92.8) and 2025 (score: 95.4)}

\cvbullet{Winner of the prize for young researchers \textit{Premio Giovani 2023 – Sezione Ricerca} organized by the Italian Society of Photogrammetry and Topography SIFET during the congress \textit{65° Convegno Nazionale SIFET}, with the contribution \textit{Monitoraggio 4D ad alta frequenza di ghiacciai alpini tramite camere time-lapse a basso costo e Deep Learning Structure-from-Motion}.
}

\cvbullet{Finalist in the \href{https://blogs.egu.eu/geolog/2024/04/08/egu24-photo-competition-finalists-who-will-you-vote-for/}{EGU2024 Photo Competition}}

% ---- Other academic merits -----
\cvsection{Other Key Academic Merits}

% --- Conference Presentations ---
\vspace{4pt}
\textbf{Presentations in Scientific Conferences}
\small
\begin{itemize}[nosep]
    \item \textbf{2025:} EGU - Wien (oral) \href{https://meetingorganizer.copernicus.org/EGU25/EGU25-15978.html}{[Abstract]}{,} Alpine Glaciological Meeting - Innsbruck (poster) \href{https://www.geo.uzh.ch/en/units/rse/news/alpine-glaciology-meeting-2024.html}{[Abstract]}.

    \item \textbf{2024:} EGU - Wien (oral) \href{https://meetingorganizer.copernicus.org/EGU24/EGU24-16412.html}{[Abstract]}.

    \item \textbf{2023:} EGU - Wien (oral) \href{https://meetingorganizer.copernicus.org/EGU23/EGU23-10018.html}{[Abstract]}{,} ISPRS Geospatial Week - Cairo (oral) \href{https://doi.org/10.5194/isprs-archives-XLVIII-1-W2-2023-1037-2023}{[Paper]}{,} VGC Dresden (oral) \href{https://doi.org/10.25368/2023.194}{[Proceedings]}{,} SIFET congress - Arezzo (IT) (oral) \href{https://www.sifet.org/convengo-2023}{[Link]}{,}  GeoAI - Torino (IT) (oral) \href{https://sites.google.com/view/geo-ai-2023/home}{[Link]}.

    \item \textbf{2022:} EGU - Wien (oral) \href{https://meetingorganizer.copernicus.org/EGU22/EGU22-7967.html}{[Abstract]}{,} ISPRS Congress - Nice (poster) \href{https://doi.org/10.5194/isprs-archives-XLIII-B2-2022-1025-2022}{[Paper]}.
\end{itemize}

% --- Peer Review / Memberships (If you have them) ---
\vspace{6pt}
\textbf{Memberships \& Peer Review}
\begin{itemize}[nosep]
    \item Reviewer for: \textit{ISPRS Journal of Photogrammetry and Remote Sensing}, \textit{Remote Sensing
              (MDPI)}, \textit{Drones (MDPI)}, \textit{Earth Surface Dynamics}, \textit{Geoscientific
              Instrumentation, Methods and Data Systems}

    \item Member of: EGU (European Geosciences Union), SIFET (Italian Society of Photogrammetry and
          Topography).
\end{itemize}
\normalsize

% ----- Additional information -----
\cvsection{Additional information}

% \cvbullet{Languages: Italian (native), English (C1 -- IELTS 2019)}
\cvbullet{UAS License: EASA A2 Open Category with Critical Scenario authorization}

\cvbullet{Driver's license: B}

\cvbullet{Professional qualification: Italian \textit{Esame di Stato} for civil and environmental engineers}

\cvbullet{Volunteer Experience:
    \begin{itemize}[nosep]
        \item Protect our Winters - POW (Climate advocacy, 2025--now);
        \item Apwoyo (Homeless support, 2023--2024);
        \item AGESCI Scout Chief (Education, 2017--2023);
        \item Operazione Mato Grosso (Charity, 2016--2017);
        \item Libera (Anti-mafia association, 2012--2016).
    \end{itemize}
}

%=======================================================================%
%   FOOTER (Date and Consent)
%=======================================================================%
\vfill
\begin{center}
    \textit{\footnotesize According to Regulation of the European Parliament 679/2016, I hereby express my consent to process and use my data in this CV and application for recruiting purposes.}\\
    \vspace{4pt}
    \textcolor{bgcol}{\hrulefill} \\
    \vspace{2pt}
    \textcolor{bgcol}{\footnotesize Francesco Ioli --- CV updated: \today}
\end{center}

\end{document}
